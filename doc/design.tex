\documentclass[12pt]{article}
\usepackage{graphicx}
\title{z-fight-club design document}
\author{Joe Doliner \and Jordan Lewis \and Nick Russo \and Matthew Steffen \and Youlian Simidjiysky}

\begin{document} \maketitle
\begin{abstract}
This document contains the high-level and component-level design of
z-fight-club's project for CMSC 23800.
\end{abstract}
\section{High-Level Design}
\subsection{Game Mechanics}
This is a racing game. As such, an individual race will be structured as
an n-lap competition of m cars, where the goal is to cross the finish line
before any other car.

A set of races will consist of multiple races of the same or different tracks,
where the ranking of each race decides the placing of cars behind the starting
line of the next one. In an m-car race, the nth placing car gets m - n points.

Optionally, cars will have redeployable grappling hooks that can be used to
temporarily anchor onto a piece of terrain to tighten a tight, high-speed turn.
\subsection{Graphics/Art direction}

\subsection{Component Architecture}
\includegraphics[scale=0.2]{game.png}

\subsection{Group organization}
\subsubsection{Communication}
Communication is done primariliy over the z-fight-club@mailman.cs.uchicago.edu maling list and the irc channel \#z-fight-club on irc.freenode.net.  We also meet in person immediately after class (5:45pm Tuesday Thursday).  We are using a git repository hosted at github.com/jordanlewis/z-fight-club for version control.  Group decisions are made based upon, *ahem*, ``rough consensus and running code.''
\subsubsection{Group Responsibilities}
Everyone is taking part in general design discussions and overall project maintenance/documentation.  For the time being, we have assigned pairs of people to each major module.  The assignments are as follows:
\begin{itemize}
\item Agents (AI):  Joe Doliner \& Jordan Lewis
\item Graphics:  Joe Doliner \& Matthew Steffan
\item Networking:  Nick Russo \& Youlian Simidjiyski 
\item Physics:  Jordan Lewis \& Youlian Simidjiyski
\item Sound:  Nick Russo \& Matthew Steffan
\end{itemize}
\subsection{Coding Style} A coding style document (style.txt) is maintained internally, and is inserted as plaintext at the end of this document.

\section{Component-Level Design}
Each component has a list of its responsibilities and an interface. These are
enumerated for each component below.
\subsection{Scheduler}
The scheduler is the top-level component of the system. It is responsible for
the main loop of the game, and for making sure that each component is run
when it needs to be without taking up more time than is it available for it.
So, the scheduler needs to tell physics, AI, controls, networking, and graphics
to conditionally update based on need every game loop.
\subsection{World}
The world is the state of the system. It is a passive component that receives
and handles requests to get or set different state attributes. Some examples
of state that belongs to the world component are car
position/velocity/orientation vectors, lighting information, camera information,
and score variables if we end up having those. This component along with physics
could possibly be the same as the Engine component in the above diagram.
\subsection{Physics}
The physics component is the engine that updates the world from its current
state to a new state, by simulating how all of the objects in the world
interact over some dt. It is responsible for receiving update requests from the
scheduler, calculating how the world looks after a time delta fom the current
state, and updating the world to that new state.
\subsubsection{Interface}
\paragraph{Called by physics}
The only input the physics subsystem requires in order to calculate the next
state of the world is the current state of the world. It also must write out
the result of its calculations. Therefore, it needs the following functions:
\begin{itemize}
\item world::poll()
\item world::set()
\end{itemize}
\paragraph{Offered by physics}
The physics engine need offer only one function, namely the simulate function
that steps the world from the current state to the next state for some timestep
dt. Optionally, simulation functions that might be useful for another component
(ex. AI trying to predict the next position of a car) could be exported by
the physics class as well, but it is unclear whether such general functions
belong in the physics class or in some external physics utility library.

\begin{itemize}
\item simulate(world, dt)
\end{itemize}

\subsection{Agent}
Agents are the in-game entities corresponding to cars. They contain the
physical parameters of the car, such as position, velocity, orientation, maximum
accleration, or turning radius. These parameters are alterable only by the
physics layer. They also contain steering parameters, which consist of a linear
acceleraton and a rotational velocity. Their movements are controlled by setting
these steering parameters.

\subsubsection{Interface}

\paragraph{Offered by agent}
\begin{itemize}
\item setSteering(SteerInfo &)
\item getKinematic(Kinematic *)
\end{itemize}
The getKinematic function returns a struct of physical read-only parameters
suitable for a component which wishes to learn about what the car is currently
doing, such as AI.

\subsection{AI}
The AI component serves as the controlling input to AI agents. It is responsible
for analyzing game state and making control decisions for each individual AI
agent currently instantiated in the syste, and sending those decisions to
each corresponding agent. The AI component is split up into an AIController
class, which is in charge of controlling an individual agent, and an AIManager
class, which is in charge of keeping track of AIControllers.

\subsubsection{Interface}

\paragraph{Offered by AI}
\begin{itemize}
\item AIManager::control(Agent)
\item AIManager::release(Agent)
\item AIManager::run()
\item AIController::run()
\end{itemize}

\subsection{Input}
The input component serves as the controlling input to human agents. It
is responsible for sampling (or being notified of?) user input, translating
that digital input (i.e. pressing forward? pressing left?) into analog
steering information suitable for giving to an Agent, and sending that steering
information to input-controlled agents.
\subsection{Networking}
The networking component is responsible for listening for network input and
doing the right thing with this input. The right thing may include world
synchronization and network agent control.  Networking will be done in the client-server fashion, with one central server holding a ``master'' copy of the world, while all clients attempt to remain in synch with the server.  The server periodically propagates the world data structure to all connected clients, and the clients update their own ``world'' data structures to approximate the server through a dead reckoning process.

\subsubsection{Implementation}

Networking will be done using the enet library, a networking library that runs over UDP that attempts to find a compromise between UDP's speed and TCP's reliability and feature set.

\paragraph{Server}
The server accepts incoming connections, and periodically recieves the world state of all connected clients.  The server uses these world states to run its own internal instance of the game, and the server periodically propagates the world state of its internal instance to all connected clients.  

%The only interesting thing server-side is that when the server calculates its internal state, it does not immediately apply all changes called by the agent.  Instead, it introduces some internal latency (equal to the expected one-way travel time from the server to the clients) in order to make the clients' dead-reckoning simulations more accurate. <-- Might implement this, need to think more about it.

\paragraph{Client} The client periodically propagates its state to the server, and periodically recieves updated state information from the server.  The client will always be some number of game cycles ahead of the most recently recieved server frame.  When the client recieves a server frame, the client repeatedly runs its physics simulation (which, in our case, should be fairly quick) upon the world frame in order to approximate the current state of the world.  So long as the approximate server state is sufficiently close to the client state, the fields in the world are linearly nudged towards the approximate server state over the next few frames.  If the approximate server state is too far from the client state, we declare the client state unsalvageable and we replace it with the approximate server state.  The client will also artificially introduce a latency into its own user's commands to better approximate the server.
 
\subsubsection{Interface}

\paragraph{Called by Networking}
The networking function needs access to the physics simulation in order to perform its dead reckoning process.  It must also be able to fetch the current state of the world and set the current state of the world.  Therefore, networking needs the following functions:
\begin{itemize} 
\item world::poll()
\item world::set()
\item physics::simulate()
\end{itemize}

\paragraph{Offered by Networking}
Networking needs to export functions allowing us to initialize a server and connect to or disconnect from a server.  Otherwise, networking should not be called by any other modules, as the networking module automatically and periodically updates the world state on its own.
\begin{itemize}
\item Server::Server(int addr, int port, int max\_connections)
\item Client::connect(int addr, int port)
\item Client::disconnect()
\end{itemize}

\subsection{Graphics}
The graphics component is the primary output of the system. It provides the
visual feedback necessary for the player to see into the game world and observe
how her input affects it. It is responsible for translating the state given
by the world component into an on-screen representation in an efficient manner.
\subsection{Sound}
The sound component translates events such as crashes, explosions, or engine
activity into audio output, and may also keep an ambient music track playing.
It fields requests like "play the crash noise" or "stop revving car C's engine"
and mixes and outputs corresponding sounds as it sees fit.

\section{Style}
This is transcribed directly from doc/style.txt.

\input{"style.txt"}

\end{document}
