\documentclass[12pt]{article}
\usepackage{graphicx}
\title{z-fight-club design document}
\author{Joe Doliner \and Jordan Lewis \and Nick Russo \and Matthew Steffen \and Youlian Simidjiysky}

\begin{document} \maketitle
\begin{abstract}
This document contains the high-level and component-level design of
z-fight-club's project for CMSC 23800.
\end{abstract}
\section{High-Level Design}
\subsection{Game Feel}
This is a racing game. What is it like? Does it have grappling hooks?
\subsection{Component Architecture}
\includegraphics[scale=0.2]{game.pdf}

\section{Component-Level Design}
Each component has a list of its responsibilities and an interface. These are
enumerated for each component below.
\subsection{Scheduler}
The scheduler is the top-level component of the system. It is responsible for
the main loop of the game, and for making sure that each component is run
when it needs to be without taking up more time than is it available for it.
So, the scheduler needs to tell physics, AI, controls, networking, and graphics
to conditionally update based on need every game loop
\subsection{World}
The world is the state of the system. It is a passive component that receives
and handles requests to get or set different state attributes. Some examples
of state that belongs to the world component are car
position/velocity/orientation vectors, lighting information, camera information,
and score variables if we end up having those. This component along with physics
could possibly be the same as the Engine component in the above diagram.
\subsection{Physics}
The physics component is an intermediary between agents and the world. It is
responsible for fielding movement requests from agents, ensuring that they are
legal requests, calculating how said requests modify the world, and finally
updating the world to match the calculation. It is also responsible for
receiving update requests from the scheduler, calculating how the world looks
after a time delta fom the current state, and updating the world to that new
state.
\subsection{Agent}
Agents are the in-game entities corresponding to car drivers. They come in
two or three categories: AI agents, human agents, and possibly network agents.
They are responsible for collecting input from different sources depending on
their category, interpreting these inputs into actions suitable to tell the
physics component, and issuing these actions to the physics component.
\subsubsection{AI}
The AI component serves as the controlling input to AI agents. It is responsible
for analyzing game state and making control decisions for each individual AI
agent currently instantiated in the syste, and sending those decisions to
each corresponding agent.
\subsubsection{Controls}
The controls component serves as the controlling input to human agents. It
is responsible for sampling (or being notified of?) user input and handing it
to the corresponding agent.
\subsection{Networking}
The networking component is responsible for listening for network input and
doing the right thing with this input. The right thing may include world
synchronization and network agent control.
\subsection{Graphics}
The graphics component is the primary output of the system. It provides the
visual feedback necessary for the player to see into the game world and observe
how her input affects it. It is responsible for translating the state given
by the world component into an on-screen representation in an efficient manner.
\subsection{Sound}
The sound component translates events such as crashes, explosions, or engine
activity into audio output, and may also keep an ambient music track playing.
It fields requests like "play the crash noise" or "stop revving car C's engine"
and mixes and outputs corresponding sounds as it sees fit.

\end{document}
